\documentclass[11pt]{article}
\usepackage{xltxtra}
\usepackage{polyglossia}
\setdefaultlanguage[spelling=modern]{russian}
%\setmainfont[Mapping=tex-text]{DejaVu Sans}
%\setmainfont[Mapping=tex-text]{Liberation Sans}
\setmonofont[Mapping=tex-text]{DejaVu Sans Mono}
\setmainfont[Mapping=tex-text]{Linux Libertine O}
%\setmonofont[Mapping=tex-text]{Liberation Mono}

\usepackage{verbatim}
\usepackage{tabularx}
\usepackage{float}
\usepackage{url}

\usepackage{indentfirst}

\usepackage{algorithm}
\usepackage{algorithmicx}
\usepackage{algpseudocode}

\usepackage[a4paper]{geometry}
\geometry{left=25mm}
\geometry{right=25mm}
\geometry{top=25mm}
\geometry{bottom=15mm}

\usepackage{listings}

\setlength{\parindent}{10mm}
\makeatletter
% Переопределение команды секции
\renewcommand{\section}{\@startsection{section}{1}%
{\parindent}{3.25ex plus 1ex minus .2ex}%
{1.5ex plus .2ex}{\bfseries\large\uppercase}}

% Переопределение команды подсекции
\renewcommand{\subsection}{\@startsection{subsection}{2}%
{\parindent}{3.25ex plus 1ex minus .2ex}%
{1.5ex plus .2ex}{\bfseries}}
\makeatother

%\newcommand{\msection}[1]{\section[#1]{#1}}
%\newcommand{\msection}[1]{\section[#1]{\uppercase{#1}}}

%opening
%\title{}
%\author{Филиппов Алексей Николаевич}

\newcommand{\titletext}{
\Large Лабораторная работа №2\\*
по дисциплине <<Методы организации нейронных сетей>>\\*
на тему <<Перцептроны>>
}

\newcommand{\myvariant}{16}
\newcommand{\mygroup}{6-78-11}

\newcommand{\includepicture}[3]{
\begin{figure}[H]
\begin{center}
\leavevmode
%\large{\textbf{#2}}
\includegraphics[width=#3\textwidth]{#1}
\end{center}
\caption{#2}
\end{figure}
}

\lstset{ %
language=C++,                   % the language of the code
basicstyle=\tiny,               % the size of the fonts that are used for the code
numbers=left,                   % where to put the line-numbers
numberstyle=\footnotesize,      % the size of the fonts that are used for the line-numbers
stepnumber=2,                   % the step between two line-numbers. If it's 1, each line
                                % will be numbered
numbersep=5pt,                  % how far the line-numbers are from the code
%backgroundcolor=\color{white},  % choose the background color. You must add \usepackage{color}
showspaces=false,               % show spaces adding particular underscores
showstringspaces=false,         % underline spaces within strings
showtabs=false,                 % show tabs within strings adding particular underscores
frame=single,                   % adds a frame around the code
tabsize=2,                      % sets default tabsize to 2 spaces
captionpos=b,                   % sets the caption-position to bottom
breaklines=true,                % sets automatic line breaking
breakatwhitespace=false,        % sets if automatic breaks should only happen at whitespace
title=\lstname,                 % show the filename of files included with \lstinputlisting;
                                % also try caption instead of title
escapeinside={\%*}{*)},         % if you want to add a comment within your code
morekeywords={*,...}            % if you want to add more keywords to the set
}

\begin{document}

\begin{titlepage}
\newpage

\begin{center}
МИНОБРНАУКИ РОССИИ\\*
Федеральное государственное бюджетное образовательное\\*
учреждение высшего профессионального образования\\*
«Ижевский государственный технический университет»\\*
Факультет «Информатика и вычислительная техника»\\*
кафедра «Программное обеспечение»
\vspace{1cm}
\end{center}

\vspace{8em}

\begin{center}
\titletext
\end{center}

\vspace{2.5em}

\begin{center}
Вариант №\myvariant
\end{center}

\vspace{6em}

\begin{flushleft}
Выполнил\\*
Студент группы \mygroup \hfill Филиппов А.Н.\\*
\vspace{1.5em}
Принял \hfill Коробейников А.В.\\*
\end{flushleft}

\vspace{\fill}

\begin{center}
Ижевск 2013
\end{center}

\end{titlepage}


\tableofcontents
\pagebreak

\section{Постановка задачи}

\subsection{Задание 1}
Реализовать с помощью перцептрона логическую функцию XOR.

\subsection{Задание 2}
Реализовать при помощи перцептрона распознавание изображений букв,
заданных в виде набора пикселей на поле ввода фиксированного размера.
Обучение произвести по дельта-правилу.

\section{Архитектура приложения}
Приложение состоит из двух частей:
\begin{enumerate}
    \item Библиотека, реализующая примитивы для работы с нейронными сетями.
    \item Графический интерфейс пользователя.
\end{enumerate}

\section{Реализация}

\newcommand{\includeclass}[2]{
\subsubsection{#2}
\lstinputlisting{include/#1.hpp}
\lstinputlisting{src/#1.cpp}
}

\subsection{Нейрон и нейронная сеть}
\includeclass{Neuron}{Нейрон}
\includeclass{NeuralLayer}{Слой нейронной сети}
\includeclass{NeuralNetwork}{Нейронная сеть}

\subsection{Обучение}
\includeclass{Supervisor}{Учитель}
\includeclass{DeltaRuleSupervisor}{Учитель, обучающий по дельта-правилу}

\subsection{Задание 1}
Функция XOR реализуется следующим перцептроном:
\includepicture{Task1NeuralNetwork}{Нейронная сеть}{1}

В данном случае необходимо как минимум два слоя, так как функция не является линейно-разделимой.

\subsection{Задание 2}
Для реализации распознавания букв была строится однослойная нейронная сеть,
содержащая столько же нейронов, сколько символов содержит распознаваемый алфавит.
Входным вектором нейросети является таблица пикселов, содержащая образ буквы.
В качестве функции активации используется пороговая (возвращает единицу если вход не меньше нуля).

Для обучения сети используется дельта-правило. Данный метод получает в качестве аргументов
ссылку на нейронную сеть и тестовый пример. После вычисления результата, который даёт нейронная сеть
на входном векторе, производится коррекция весов с использованием выходного вектора-цели (в данной задаче вектор имеет размер 1).

\section{Тестовые примеры}
\includepicture{Task1100}{Задание 1, x = 0, y = 0}{1}
\includepicture{Task1101}{Задание 1, x = 0, y = 1}{1}
\includepicture{Task1110}{Задание 1, x = 1, y = 0}{1}
\includepicture{Task1111}{Задание 1, x = 1, y = 1}{1}
\includepicture{Task12ф}{Задание 2, буква ф}{1}
\includepicture{Task12и}{Задание 2, буква и}{1}
\includepicture{Task12л}{Задание 2, буква л}{1}
\includepicture{Task12п}{Задание 2, буква п}{1}
\includepicture{Task12о}{Задание 2, буква о}{1}
\includepicture{Task12в}{Задание 2, буква в}{1}

\end{document}
